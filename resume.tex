%% start of file `template.tex'.
%% Copyright 2006-2013 Xavier Danaux (xdanaux@gmail.com).
%
% This work may be distributed and/or modified under the
% conditions of the LaTeX Project Public License version 1.3c,
% available at http://www.latex-project.org/lppl/.


\documentclass[11pt,a4paper,sans]{moderncv}        % possible options include font size ('10pt', '11pt' and '12pt'), paper size ('a4paper', 'letterpaper', 'a5paper', 'legalpaper', 'executivepaper' and 'landscape') and font family ('sans' and 'roman')

% modern themes
\moderncvstyle{banking}                            % style options are 'casual' (default), 'classic', 'oldstyle' and 'banking'
\moderncvcolor{blue}                                % color options 'blue' (default), 'orange', 'green', 'red', 'purple', 'grey' and 'black'
%\renewcommand{\familydefault}{\sfdefault}         % to set the default font; use '\sfdefault' for the default sans serif font, '\rmdefault' for the default roman one, or any tex font name
%\nopagenumbers{}                                  % uncomment to suppress automatic page numbering for CVs longer than one page

\definecolor{color1}{rgb}{0.22,0.45,0.70}

% character encoding
\usepackage[utf8]{inputenc}                       % if you are not using xelatex ou lualatex, replace by the encoding you are using
%\usepackage{CJKutf8}                              % if you need to use CJK to typeset your resume in Chinese, Japanese or Korean

% adjust the page margins
\usepackage[scale=0.75]{geometry}
%\setlength{\hintscolumnwidth}{3cm}                % if you want to change the width of the column with the dates
%\setlength{\makecvtitlenamewidth}{10cm}           % for the 'classic' style, if you want to force the width allocated to your name and avoid line breaks. be careful though, the length is normally calculated to avoid any overlap with your personal info; use this at your own typographical risks...

\usepackage{import}

% personal data
\name{Fabio}{Silva}
%\title{CV}                               % optional, remove / comment the line if not wanted
%\address{my address, line 1, line 2, line 3, postcode}{}{}% optional, remove / comment the line if not wanted; the "postcode city" and and "country" arguments can be omitted or provided empty
\phone[mobile]{+55~(83)~9~8858~2685}                   % optional, remove / comment the line if not wanted
%\phone[fixed]{01234 123456}                    % optional, remove / comment the line if not wanted
%\phone[fax]{+3~(456)~789~012}                      % optional, remove / comment the line if not wanted
\email{fabio.fernando.osilva@gmail.com}                               % optional, remove / comment the line if not wanted
%\homepage{ffosilva.github.io}                         % optional, remove / comment the line if not wanted
\social[github]{github.com/ffosilva}                         % optional, remove / comment the line if not wanted
\social[linkedin]{linkedin.com/in/ffosilva}                         % optional, remove / comment the line if not wanted
%\extrainfo{additional information}                 % optional, remove / comment the line if not wanted
%\photo[64pt][0.4pt]{picture}                       % optional, remove / comment the line if not wanted; '64pt' is the height the picture must be resized to, 0.4pt is the thickness of the frame around it (put it to 0pt for no frame) and 'picture' is the name of the picture file
%\quote{Some quote}                                 % optional, remove / comment the line if not wanted

% to show numerical labels in the bibliography (default is to show no labels); only useful if you make citations in your resume
%\makeatletter
%\renewcommand*{\bibliographyitemlabel}{\@biblabel{\arabic{enumiv}}}
%\makeatother
%\renewcommand*{\bibliographyitemlabel}{[\arabic{enumiv}]}% CONSIDER REPLACING THE ABOVE BY THIS

% bibliography with mutiple entries
%\usepackage{multibib}
%\newcites{book,misc}{{Books},{Others}}
%----------------------------------------------------------------------------------
%            content
%----------------------------------------------------------------------------------
\begin{document}
%\begin{CJK*}{UTF8}{gbsn}                          % to typeset your resume in Chinese using CJK
%-----       resume       ---------------------------------------------------------
\makecvtitle

%\small{Computer Science student completing the final year of a bachelor's degree. Passionate about Cyber Security, DevOps, Cloud Computing, music and travelling.}

\section{Experience}

\vspace{6pt}

\begin{itemize}

\item{\cventry{Jan 2017 -- Mar 2019}{Research Assistant \& Software Developer}{Distributed Systems Laboratory}{Campina Grande, Brazil}{}{\vspace{3pt}Development of an ecosystem of cloud facilities characterized by superior security guarantees, providing protection from attacks by privileged users (e.g. the cloud provider or the system administrator) and software (e.g. the hypervisor). Protection relies on new security extensions recently introduced into commercially available off-the-shelf CPUs.\\
		{\color{gray} Tech Stack:} Docker, SCONE, OpenStack, Intel SGX, C++, MongoDB, Python, Apache Kafka, Git.}}

\vspace{6pt}

\item{\cventry{Dec 2017 -- Jan 2018}{Research Assistant \& Software Developer, Intern}{Technische Universität Dresden}{Dresden, Germany}{}{\vspace{3pt}Improve the Python support and performance on SCONE. The solution includes modifications on CPython Source Code (C) and enable the use of Intel Math Kernel Library inside SGX enclaves.\\
		{\color{gray} Tech Stack:} Docker, SCONE, Intel SGX, C, Python, musl libc, PySpeed, PyPy, Git.}}

\vspace{6pt}

\item{\cventry{Jan 2016 -- Dec 2016}{Software Developer}{PrivIoT}{Campina Grande, Brazil}{}{\vspace{3pt}Researched good practices for developing Internet of Things applications that respect people's privacy, building several applications that serve as case studies for research, and validation of our recommendations for software developers.\\
{\color{gray} Tech Stack:} MongoDB, Python, Flask, Ionic, JavaScript, AngularJS, D3.js, Git.}}

\vspace{6pt}

\item{\cventry{Aug 2014 -- Dec 2015}{Data Analyst \& Software Developer}{Analytics Lab}{Campina Grande, Brazil}{}{\vspace{3pt}Worked on analysis of 3 years of consortium of agricultural cohort data of more than 200 farmers, improving visualization of data using JavaScript frameworks and reporting the analysis to improve the consortium. In the public transportation field, worked on development of auxiliary mobile application that helped register city's bus stops and validate bus routes, bringing precision by using GPS and completing the dataset provided by our partner. \\
{\color{gray} Tech Stack:} R, D3.js, Chart.js, Android SDK, Git.}}

\vspace{6pt}

\item{\cventry{Nov 2012 -- Jul 2014}{Software Developer \& Research Assistant}{Software Practices Laboratory}{Campina Grande, Brazil}{}{\vspace{3pt}At SPLAB I worked on development of an Eclipse plug-in that allows the developer to switch Java Collections on JCF client applications source code with a few clicks. Demonstrator video: https://www.youtube.com/watch?v=PUIEPlVmI-Q\\
		{\color{gray} Tech Stack:} Java, Eclipse, Eclipse JDT, Design Patterns, Git.}}

\newpage
%\vspace{6pt}

\item{\cventry{Nov 2011 -- Feb 2012}{Web \& Android Developer, Intern}{9Ideia Digital}{Campina Grande, Brazil}{}{\vspace{3pt}Created web pages and content management systems (CMS) over Apache-MySQL-PHP stack, also developed several Android applications using Android SDK. Mandatory internship to get Certificate in Information Technology.\\
		{\color{gray} Tech Stack:} Apache, HTML, CSS, JavaScript, Java, Android SDK, jQuery, PHP, MySQL, Git.}}

\end{itemize}

\section{Voluntary Experience}

\vspace{6pt}

\begin{itemize}
	
	\item{\cventry{Nov 2015 -- Oct 2016}{Device Tree Maintainer}{CyanogenMod Project}{}{}{\vspace{3pt}Added initial support to Sony Xperia Z3 Dual (D6633) on CyanogenMod project. As device tree maintainer, I also reviewed and approved contributions made through pull requests on Gerrit Code Review system. \\
			{\color{gray} Tech Stack:} Linux Kernel, Android Build Engine, AOSP, Android Architecture, C++, Makefile, Gerrit, Git.}}

\end{itemize}

\section{Education}

\vspace{5pt}

\begin{itemize}

\item{\cventry{2018 -- 2020 (expected)}{M.Sc. Computer Science}{Federal University of Campina Grande}{Campina Grande, Brazil}{}{}}

\item{\cventry{2012 -- 2017}{B.Sc. Computer Science, \textbf{GPA: 8.18/10}}{Federal University of Campina Grande}{Campina Grande, Brazil}{}{}}

\item{\cventry{2010--- 2012}{Certificate in Information Technology, \textbf{GPA: 10/10}}{Infogenius Technical College}{Campina Grande, Brazil}{}{}}

\end{itemize}

\section{Notable Projects}

\vspace{5pt}

\begin{itemize}

\item{\cventry{}{Founder \& Developer}{RunMusic App}{}{Android Application}{\vspace{3pt}RunMusic gives to the user an innovative experience of listening to the music that mostly matches their running pace. This application won the 1st place in Entrepreneurship category of III Campus Mobile national contest.}}

\vspace{6pt}

\item{\cventry{}{Academic Project}{Java Compiler}{}{Compiler Course}{\vspace{3pt}Built a simplified Java Compiler using JFlex and CUP. The front-end of the compiler is covering most of the language. The back-end generates Assembly code for a simplified scope of the language.}}

\end{itemize}

\section{Key Skills}

\vspace{6pt}

\begin{itemize}

\item \textbf{Programming Languages:} Python, Java, JavaScript, R, C/C++.

\vspace{6pt}

\item \textbf{Server Technologies:} Node.js, Flask, PHP.

\vspace{6pt}

\item \textbf{Cloud Technologies:} Docker, OpenStack, Kubernetes.

\vspace{6pt}

\item \textbf{Others:} Git, Gerrit, Maven, Linux, SQL, MongoDB.

\end{itemize}

\section{Languages}

\vspace{6pt}

\begin{itemize}

\item{Portuguese \textit{(native)}, English \textit{(professional proficiency)}, Spanish \textit{(moderate)}.}

\end{itemize}

\section{Awards}

\vspace{6pt}
 
\begin{itemize}

\item{\textbf{2015:} Best application in Entrepreneurship Category of III Campus Mobile national contest, promoted by Instituto Embratel-Claro.}

\item{\textbf{2011:} 1st place on programming contest promoted by Infogenius Technical College.}

\end{itemize}

% Publications from a BibTeX file without multibib
%  for numerical labels: \renewcommand{\bibliographyitemlabel}{\@biblabel{\arabic{enumiv}}}% CONSIDER MERGING WITH PREAMBLE PART
%  to redefine the heading string ("Publications"): \renewcommand{\refname}{Articles}
\nocite{*}
\bibliographystyle{abbrv}
\bibliography{publications}                        % 'publications' is the name of a BibTeX file

% Publications from a BibTeX file using the multibib package
%\section{Publications}
%\nocitebook{book1,book2}
%\bibliographystylebook{plain}
%\bibliographybook{publications}                   % 'publications' is the name of a BibTeX file
%\nocitemisc{misc1,misc2,misc3}
%\bibliographystylemisc{plain}
%\bibliographymisc{publications}                   % 'publications' is the name of a BibTeX file

%-----       letter       ---------------------------------------------------------

\end{document}


%% end of file `template.tex'.
